% options are:
% PhD, MSc (choose one)
% beforeExam (to make the personal thanks invisible)
\documentclass[MSc,beforeExam]{iitcsthesis}

\usepackage{fontspec}
\usepackage{etoolbox}
\usepackage{polyglossia}
\usepackage{bidi}

\setdefaultlanguage{english}
\setotherlanguage{hebrew}
\newfontfamily\hebrewfont[Script=Hebrew]{Droid Sans Hebrew}

% For some reason, the hebrew package clashes with the amsthm package. If you need a proof environment, you can uncomment the following:
%\usepackage{amssymb} %Needed for \blacksquare. Take care not to add this package twice...
%\newenvironment{proof}[1][Proof]{\par \textbf{#1.} }{\hspace{10pt}\hfill$\blacksquare$\par}
\begin{document}

\bibliographystyle{plain}

\authorEnglish{Alex Kreimer}

\titleEnglish{Infinite Odometry}

\supervisorEnglish{The research thesis was done under the supervision
of Prof./Dr.~(First Name) (Last Name) in the Computer Science Department.}

\GregorianDateEnglish{October 2008}
\JewishDateEnglish{Tishrei 5769}

\personalThanksEnglish{Write thanks to professor and others}
\financialThanksEnglish{The generous financial support of the Technion (and someone else if you got a scholarship) is gratefully acknowledged.}

\maketitleEnglish

% The English abstract should be 200-500 words long.
\abstractEnglish In this work we revisit a problem of visual odometry
(VO). VO is the process of estimating the egomotion of the camera by
examining the changes that the motion induces on the images made by
it. The approach we propose exploits scene structure typical for that
seen by a moving car and is suitable for use in both monocular and
stereo settings. We recover rotation and translation separately, thus
dealing with two separate (smaller) problems. The rotation is
estimated by means of infinite homography which benefits from
additional data in stereo setting but may also be used in the
monocular setting. We start with an initial estimate and then refine
it using iterative procedure. After the rotation is compensated for
the translation is found by means of 1-point algorithm in stereo
setting and epipole computation for pure translational motion in
monocular setting. We evaluate our algorithm on the
KITTI~\cite{geiger2013vision} dataset.

\abbreviationsAndNotationsEnglish

\begin{tabular}{lcl}
$VO$ & --- & Visual odometry\\
$GPS$ & --- & Global Positioning System\\
$IMU$ & --- & Inertial Navigation Unit\\

\end{tabular}

%Or, if your tables are long, ``\usepackage{longtable}'' at the beginning, and then  ``\begin{longtable}[l]{lcl} ... \end{longtable}''

\chapter{Introduction}

Recovery of the agent (vehicle, robot, etc.) motion is a basic
building block for many higher-level tasks. Various methods may be
used to recover this motion, e.g., wheel odometry, global positioning
system (GPS), inertial measurement units (IMU), laser odometry and
visual odometry. Contrary to wheel odometry, VO is not affected by
wheel slip and usually provides a more accurate trajectory
estimate. Contrary to GPS VO does not depend on satellite signal
availability which may be a concern in urban canyons and other
GPS-denied environments. All these makes VO a viable alternative to
agent egomotion estimation. Of course, VO has demands of its own,
e.g., sufficient illumination, image overlap, static scene with enough
texture.

\section{Background}

\section{Our Approach}
\subsection{Contributions}
\section{Thesis Outline}

\chapter{Related Work}


\section{A section}

Some stuff.

\section*{Appendix}
\addcontentsline{toc}{section}{Appendix}
For some reason, you need to manually add stared sections to the the index file.

\bibliography{thesis}{}

\authorHebrew{\texthebrew{השם שלך}}

\titleHebrew{\texthebrew{כותרת התזה}}

\supervisorHebrew{\texthebrew{המחקר נעשה בהנחיית פרופ'/דר' )פרטי( )משפחה( בפקולטה למדעי המחשב.}}

\personalThanksHebrew{\texhebrew{תודות אישיות למנחים ולאחרים}}

\financialThanksHebrew{
\begin{hebrew}
אני מודה לטכניון )ולגוף שנתן מלגה אם קבלת( על התמיכה הכספית הנדיבה בהשתלמותי.
\end{hebrew}
}

\JewishDateHebrew{\texthebrew{תשרי ה'תשס"ט}}

\GregorianDateHebrew{\texthebrew{אוקטובר \beginL2008\endL}}

\maketitleHebrew

% 1000-2000 words
\abstractHebrew
\begin{hebrew}

צריך לכתוב את התקציר בעברית

\end{hebrew}
\finishedHebrew


% Actually, I found it much easier to have each chapter in a file of its own. Except for the Hebrew part, I put it all together in this example just to make things clear.

\end{document}
