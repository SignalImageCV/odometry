\documentclass[10pt]{article}         %% What type of document you're writing.

\usepackage{amsmath,amsfonts,amssymb,mathtools}   %% AMS mathematics macros
\usepackage{graphicx}
\usepackage{caption}
\usepackage{subcaption}

\DeclarePairedDelimiter\abs{\lvert}{\rvert}%
\DeclarePairedDelimiter\norm{\lVert}{\rVert}%

\title{Visual Odometry: Literature Survey}
\author{Alex Kreimer}

\begin{document}

\maketitle

\begin{abstract}
\end{abstract}

\subsection{A way to parameterize rotations}
~\cite{schmidt2001using} proposes a method to use quaternions in an
unconstrained nonlinear optimization.  Quaternions representing
rotation have four elements but only three degrees of freedom, since
they have to be of norm one.  This constraint has to be taken into
account when applying e.g. Levenberg-Marquardt algorithm.  One of the
ways to address this issue is to use appropriate parameterization
(others are a projection step and Lagrange multipliers). Well known
parameterizations are Euler angles and axis-angle representation.

~\cite{hornegger1999representation} call a parameterization fair if it
does not introduce more numerical sensitivity than inherent to the
problem itself.  This is guaranteed, if any rigid transformation of
the space to be parameterized results in an orthogonal transformation
of the parameters.  Both axis-angle and quaternion parameterizations
are fair, while Euler angles is not.

Authors search for a parameteriation that:
\begin{enumerate}
\item is minimal, i.e. uses only three parameters
\item the three parameters may be changed arbitrarily by the optimization algorithm
\item the resulting quaternion has always norm 1.
\end{enumerate}

This new approach is based on the observation that all quaternions of
norm-1 lie on the unit sphere in $\mathbb{R}^4$.  The authors use the
shortest connection between two points on a sphere, i.e. a great
circle.  For describing a movement on the sphere starting at
$\mathbf{h_0}$ they use a vector $v_4$ lying in the tangential
hyperplane that touches the sphere at $\mathbf{h_0}$. This hyper-plane
is a subspace of $\mathbb{R}^4$, thus vectors in this plane may be
represented as 3-vectors with respect to a plane-local coordinate
frame.

Experiments are made on a synthetic (small) data-set.  The authors
perform bundle adjustment and compare their approach with axis-angle
representation.  The conclusion is that this representation performs
better for rotations, for transnational motion both method are
approximately equal.

\bibliography{survey}{}
\bibliographystyle{plain}
\end{document}
